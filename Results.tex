\section{Results}

\subsection{ABL Precursor Results}
The temperature profile of the precursor ABL simulation shows the capping inversion that limits boundary layer growth and vertical motion above 750 m.  Figures XX show the fully developed flow fields for neutral and unstable ABL's after they've been allowed to spin up for 12000 seconds.  The precursor simulations show realistic coherent turbulent structures that are crucial in generating realistic load distributions.  The different stability cases also exhibit realistic vertical mixing and shear profiles, with the unstable case showing a better mixed boundary layer.

\subsection{ABL Actuator Line Results}
After fully developing the atmospheric boundary layer in the absence of wind turbines, we insert a single actuator line turbine and run the simulation for another 250 seconds.  The results lead to several interesting insights.  First, there is nontrivial flow leakage at the root of the blades that persists for about two rotor diameters downstream.  This region where the blade airfoil properties are still discernable is the "near-wake" region\cite{sanderse_review_2011}.  Further downstream in the "far-wake" region the blade-specific effects have diffused into an amorphous wake, but the velocity deficit that is characteristic of the far-wake is present for eight to ten rotor diameters.  Figures XX show average velocity profiles and Figures XX show instantaneous velocity profiles at 250 seconds.  The instantaneous velocities also show horizontal and vertical meandering of the wake that is likely caused by instabilities in the boundary layer.

Interestingly, we also observe numerical artifacts in the CFD results located at the transition between grid refinement regions.  Figure XX shows the average velocity at hub height for the unstable ABL with the actual grid cells overlaid on the flow field.  Clearly, the artifacts are generated at the abrupt transition between adjacent refinement levels.  Conversations with the SOWFA creators, Matt Churchfield and Sang Lee, suggest that a spatial filter or mesh smoothing function could help alleviate these anomalies.



\subsection{OBL Results}

