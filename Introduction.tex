\section{Introduction}
High fidelity modeling of fluid flow is crucial in calculating mechanical loads, power production, and wake behaviors of wind and ocean turbines.  The wind industry has suffered from unexpected gearbox failures and underperformance, and recent publications attribute these effects to poor characterization and prediction of the turbulent flow fields experienced by wind turbines.  To alleviate these effects, CFD calculations must be able to model realistic atmospheric and ocean boundary layers flows that occur in wind and ocean turbine installations.  In particular, these CFD simulations must account for turbulent coherent structures that are the primary cause of fatigue damage to wind turbine blades and drivetrains\cite{kelley_impact_2005,lee_atmospheric_2011}.  Additionally, these CFD simulations must include thermal stability effects because stability has been shown to impact the power production of single turbines and affect the wake behavior from arrays of turbines\cite{wharton_atmospheric_2012,wharton_assessing_2011}.  More accurate modeling of these turbulent flow fields will lead to improved wind turbine reliability and more energetic wind farm arrays, ultimately lowering the cost of renewable energy.

In response to these concerns, there have been several new developments in computational modeling of wind farms.  In particular, large eddy simulations (LES) have been used to  model wind turbines and their arrays in realistic ABL flows\cite{calaf_large_2010,lu_large-eddy_2011,porte-agel_large-eddy_201,wu_large-eddy_2010,sanderse_review_2011}.  In this study we take an LES approach to generating realistic neutrally stable and convective ABL's, and extend that framework to a stable ocean boundary layer (OBL).  This approach builds on the Simulator for Offshore Wind Farm Applications (SOWFA) toolset developed by Matt Churchfield and Sang Lee at the National Wind Technology Center\cite{SOWFA}.  This toolset uses OpenFOAM, an open source finite volume CFD program\cite{OpenFOAM}, to generate precursor boundary layer simulations exhibiting realistic turbulent structures.  Actuator line models of wind and ocean turbines are then inserted into the fully developed boundary layer flows following the work of Churchield and Troldborg\cite{churchfield_large-eddy_2012,churchfield_numerical_2012,troldborg_actuator_2007}.  Our results show interesting differences in the wake behaviors of ocean and atmospheric turbines that have important implications for array optimization.  Additionally, we discuss the emergence of numerical artifacts and comment on their impact on the choice of LES grid cell size.
