\section{Conclusion and Future Work}
This study replicated existing work on LES simulations of wind turbines, and confirmed many previous findings.  Specifically, our ABL simulations showed the wakes dissipating sooner in the unstable case, which fits with other published findings.  We also replicated anomalous results due to grid refinement that was originally observed by the creators of the SOWFA toolset.  We have extended the SOWFA toolset to ocean boundary layers and demonstrated much greater periodic motion of the wake in the vertical direction.  Our ocean simulation also shows that the artifacts disappear when grid resolution step changes are removed.

There is a clear need for future work on LES modeling of wind turbines, and we believe there are many opportunities to improve the numerical methods deployed in SOWFA.  With the simplicity of the domain, the tendency towards smooth functions and the periodic boundary condition, spectral methods and wavelet decompositions would likely result in a substantial performance gain. Additionally, our simulations used the backward Euler method for time steps, where a performance would likely be gained in using the Crank Nicholson or Runge Kutta methods.

Future work on atmospheric turbine modeling will likely focus on including aeroelastic tools such as the NWTC's FAST aeroelastic simulator, as well as expanding the actuator line tools to larger turbine arrays in more complicated terrain.  SOWFA's modular nature will allow for detailed mechanical loads analysis of a single turbine, or broader resource assessment and turbine layout optimization.

Future work on this project related to ocean turbine modeling includes several possibilities. For one, the grid size used in the initial model was very coarse, with a grid cell size of 2m$^3$ in the precursor stage and 0.2m$^3$ in the turbine modeling stage. A more accurate representation of the precursor flow could be obtained by doubling or tripping the resolution. Second, work remains on better modeling the various tidal turbines currently being proposed by the private sector; the model used in this paper was adapted from a wind turbine design. And finally, work remains in modeling various ocean states, such as stable, unstable, high wind shear/turbulence, etc.



